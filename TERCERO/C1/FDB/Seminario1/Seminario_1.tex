\documentclass[a4paper,10pt]{article}
\title{FDB:Seminario 1 }
\author{Daniel Monjas Miguélez}

\usepackage[spanish]{babel}
\usepackage[utf8]{inputenc}
\usepackage{hyperref}
\usepackage{graphicx}
\graphicspath{ {images/} }

\begin{document}
\maketitle

\newpage

\tableofcontents

\newpage

\section{Etapas de la creación de una BD}
\begin{itemize}
\item Datos generales sobre una organizació concreta.

\item Datos operativos que se manejan en la organización.

\item Esquema conceptual de la base de datos.

\item Modelo lógico de la base de datos.

\item Implementación de la base de datos en un DBMS.
\end{itemize}

\textbf{Esquema conceptual de la base de datos:} organizar los datos relevantes para el funcionamiento de una empresa.

\section{Modelo Entidad/Relación}
Es un mecanismo formal para representar y manipular información de manera general y sistemática. \\

Claves de este modelo
\begin{itemize}
\item Datos:hay que analizarlos con detenimiento y realizar control de datos.

\item Convenciones:notación rigurosa y normalizada, y seguir una línea de actuación sistemática.

\item Redundancia mínima: cualquier dato o concepto debe ser modelado de manera única.
\end{itemize}

Requisitos:

\begin{itemize}
\item Reflejar fielmente las necesidades de información de una organización.

\item Ofrecer un diseño independiente del posterior almacenamiento de los datos y sus métodos de acceso.
\end{itemize}

\section{Elementos básicos del modelo}
El enfoque E-R se basa en la clasificación de los datos en:

\begin{itemize}
\item \textbf{Entidades:} objetos de nuestro interés agrupados por tipo.

\item \textbf{Relaciones:} representan las conexiones existentes entre objetos.

\item \textbf{Atributos:} características de interés de las entidades y relaciones consideradas.

\item \textbf{Dependencia existencial:} no se puede identificar un dato sin otro previo, por ejemplo, no se pueden identificar los movimientos de una cuenta bancaria sin conocer dicha cuenta.Entidad debil-> dependientes, entidad fuerte-> entidad de la que depende la debil.

\item Asociaciones o relaciones: conexión semántica entre dos o más conjuntos de entidades.

\item Cardinalidad: número máximo de entidades de un conjunto que se conecta o relaciona con una entidad de otro y vicecersa.
\end{itemize}


\end{document}