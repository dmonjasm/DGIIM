\documentclass{article}


\author{Daniel Monjas Migu\'elez}
\usepackage[utf8]{inputenc}
\usepackage[T1]{fontenc}
\usepackage[spanish,es-tabla]{babel}
\usepackage[hidelinks]{hyperref}
\hypersetup{
	colorlinks=true,
	linkcolor=blue,
	filecolor=magenta,
	urlcolor=cyan,
}

\begin{document}

\textbf{Autor:Daniel Monjas Miguélez} \\ \\

\textbf{6.- a) Una población se rige por el modelo 
\begin{equation}
p_{n+1}=10*p_n*e^{-p_n}, n \geq 0
\end{equation}
Prueba que los equilibrios son inestables.}

Definimos $f(x)=10*x*e^{-x}$ y buscamos los puntos tales que $f(x)=x$ para ello:


\begin{equation}
10*x*e^{-x}=x \Longleftrightarrow x*(10*e^{-x} - 1) = 0 \Rightarrow x=0 \> o \> x = ln(10)
\end{equation}

Una vez tenemos los puntos de equilibrio calculamos la derivada de $f(x)$ y evaluamos en dichos puntos la función:

\begin{equation}
f'(x)=10*(e^{-x}-x*e^{-x})
\end{equation}

Luego $f'(0)=10 \notin (-1,1)$ y $f'(ln(10))=1-ln(10) \notin (-1,1)$. Como al evaluar $f'(x)$ en $x=0$ y en $x=ln(10)$ se tiene que las imagenes para dichos valores de $x$ no están en $(-1,1)$ llegamos a que dichos puntos de equilibrio no son estables. \\

\textbf{6.- b) Para conseguir un equilibrio poblacional localmente asintóticamente estable (a.e), se propone vender una fracción $\alpha$ ($0 < \alpha < 1)$ de la población en cada periodo de tiempo dando lugar al modelo: 
\begin{equation}
p_{n+1}=10*(1-\alpha)*p_n*e^{-(1-\alpha)p_n}
\end{equation} 
} \\

\textbf{b) i) Encuentre el intervalo abierto (de amplitud máxima) donde elegir $\alpha$ para que esté asegurada la estabilidad asintótica local del equilibrio positivo.}

En primer lugar buscaremos los puntos fijos, por lo tanto definimos $f(x)=10*(1-\alpha)*x*e^{-(1-\alpha)*x}$  e igualamos $f(x)=x$.

\begin{equation}
f(x) = x  \Longleftrightarrow x=10*(1-\alpha)x*e^{-(1-\alpha)*x}
\end{equation}

Despejando la ecuación anterior obtenemos que $x=0$ es solución pero la descartamos porque nos preguntan por puntos de equilibrio positivos. La otra solución se obtiene al despejar:

\begin{equation}
1=10*(1-\alpha)*e^{-(1-\alpha)*x} \Longleftrightarrow ln(\frac{1}{10*(1-\alpha)}) = -(1-\alpha)*x
\end{equation}

finalmente obtenemos que el segundo punto de equilibrio dependerá del $\alpha$ y tendrá la siguiente forma 


\begin{equation}
x=\frac{ln(10*(1-\alpha))}{(1-\alpha)} \label{funcion}
\end{equation}

Ahora el objetivo será asegurar que x sea positivo, luego buscaremos la primera cota para $\alpha$. Buscamos que $x > 0$ y debemos tener en cuenta que como $0 < \alpha < 1$ se tiene que $-(1-\alpha) < 0$, por consiguiente,

\begin{equation}
x > 0 \Longleftrightarrow ln(\frac{1}{10*(1-\alpha)}) < 0 \Longleftrightarrow \alpha < \frac{9}{10}
\end{equation}

Por último el objetivo será el obtener para que valores de $\alpha$ los puntos x de la forma (\ref{funcion}), son estables. En primer lugar calculamos la derivada de $f(x)$,

\begin{equation}
f'(x)=10*(1-\alpha)*(e^{-(1-\alpha)*x}+x*e^{-(1-\alpha)*x}*(-(1-\alpha)))
\end{equation}

Ahora evaluamos la expresión (\ref{funcion}) y buscamos los valores de $\alpha$ para los que x es estable.

\begin{equation}
f'(\frac{ln(10*(1-\alpha))}{(1-\alpha)})=1-ln(10*(1-\alpha))
\end{equation}

Sabemos que para que x sea estable $f'(x) \in (-1,1)$,  por tanto

\begin{equation}
1-ln(10*(1-\alpha)) < 1 \Longleftrightarrow 10*(1-\alpha) > 1 \Longleftrightarrow \alpha < \frac{9}{10}
\end{equation}

Ahora hacemos la versión análoga para -1

\begin{equation}
1-ln(10*(1-\alpha)) > -1 \Longleftrightarrow e^2 > 10*(1-\alpha) \Longleftrightarrow \alpha > \frac{10-e^2}{10}
\end{equation}

Llegando finalmente a la conclusión de que un punto de equilibrio será estable y positivo si tiene la forma de la expresión (\ref{funcion}) y $\alpha \in (\frac{10 - e^2}{10}, \frac{9}{10})$. \\

\textbf{b) ii) Calcula el valor m\'aximo de $\alpha$ para el que la poblaci\'on de equilibrio alcanza su valor m\'aximo y el localmente estable.}

Sabemos que para que x sea punto fijo y estable $\alpha \in (\frac{10-e^2}{10},\frac{9}{10})$ y 

\begin{equation}
x=\frac{ln(10*(1-\alpha))}{(1-\alpha)}
\end{equation}

Buscaremos el m\'aximo de x en funci\'on de $\alpha$ para lo que definir\'e la siguiente funci\'on:

\begin{equation}
g(\alpha)=\frac{ln(10*(1-\alpha))}{(1-\alpha)}
\end{equation}

Se hace la derivada y se iguala a 0 para obtener puntos cr\'iticos de la funci\'on.

\begin{equation}
g'(\alpha)=\frac{-1-ln(\frac{1}{10*(1-\alpha)})}{\alpha^2-2\alpha+1}
\end{equation}

Igualando la derivada a cero se obtiene que,

\begin{equation}
g'(\alpha)=0 \Longleftrightarrow -1-ln(\frac{1}{10*(1-\alpha)})=0 \Longleftrightarrow \frac{1}{10*(1-\alpha)} = \frac{1}{e}
\end{equation}

despejando la expresi\'on anterior se obtiene que $\alpha=\frac{10-e}{10} \in (\frac{10-e^2}{10}, \frac{9}{10})$, por lo que x es estable, ahora falta por comprabar que se trata de un m\'aximo. Para ello hacemos la segunda derivada de la funci\'on,

\begin{equation}
g''(\alpha)=\frac{3+2*ln(\frac{1}{10* (1-\alpha)})}{(\alpha-1)^3}
\end{equation}

Y ahora evaluamos $g''(\alpha)$ en el punto cr\'tico que hemos obtenido, de forma que 

\begin{equation}
g''(\frac{10-e}{10})=\frac{-1000}{e^3} < 0
\end{equation}

De donde obtenemos que el punto $\alpha=\frac{10-e}{10}$ es un m\'aximo relativo de $g(\alpha)$ y adem\'as $\alpha \in(\frac{10-e^2}{10},\frac{9}{10})$ por lo que $\alpha$ sería el máximo valor en el que x alcanza su valor máximo y es localmente estable.


\end{document}